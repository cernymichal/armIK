\documentclass[czech]{pyt-report}

\usepackage[utf8]{inputenc}

\usepackage{algorithm,algpseudocode}
\makeatletter
\renewcommand{\ALG@name}{Algoritmus}
\makeatother

\graphicspath{{./images/}}

\title{FABRIK na systémech kloubů s 1 DOF}

\author{Michal Černý}
\affiliation{FIT ČVUT}
\email{cernym65@fit.cvut.cz}

\def\file#1{{\tt#1}}

\begin{document}

\maketitle

%%%%%%%%%%%%%%%%%%%%%%%%%%%%%%%%%%%%%%%%%%%%%%%%%%%%%%%%%%%%%%%%%%%%%%%%%%%%%%%%
\section{Úvod}
\label{sec:uvod}
Tato zpráva se věnuje semestrální práci předmětu BI-ZUM.21 v letním semestru 2022/23 zaměřené na použití algoritmu FABRIK\cite{bib:fabrik} pro řešení inverzní kinematiky ramena složeného z 6 kloubů s pouze jedním stupněm volnosti. Výstupem je implementace v herním enginu Unity\cite{bib:unity} a interaktivní aplikace s ramenem.

%%%%%%%%%%%%%%%%%%%%%%%%%%%%%%%%%%%%%%%%%%%%%%%%%%%%%%%%%%%%%%%%%%%%%%%%%%%%%%%%
\section{Existující řešení}
\label{sec:exitujici_reseni}
V robotice je nejčastěji využíváno numerické řešení pomocí hledání inverzí Jacobiho matic. Tato metoda produkuje ladné konfigurace, ale trpí vysokou výpočetní náročností v podobě hledání rozkladu matic. Pro aplikace v reálném čase jako 3D animace se potom využívají heuristické metody Cyclic Coordinate Descent (CCD) nebo Forward And Backward Reaching Inverse Kinematics\cite{bib:fabrik} (FABRIK).

%%%%%%%%%%%%%%%%%%%%%%%%%%%%%%%%%%%%%%%%%%%%%%%%%%%%%%%%%%%%%%%%%%%%%%%%%%%%%%%%
\section{FABRIK}
\label{sec:fabrik}
FABRIK je velmi jednoduchý iterativní algoritmus\cite{bib:fabrik}. Pokud předpokládáme posloupnost pozic $n$ kloubů $p$, kde $p_1$ je základna, jejich vzdálenosti $d$ a cílový bod $t$, jeden krok základní verze algoritmu vypadá následovně:

\begin{algorithm}[H]
\caption{Krok algoritmu FABRIK}\label{alg:fabrik}
\begin{algorithmic}[1]
\Statex \texttt{/* Dopředné natahování */}
\State $p_n = t$
\For{$i = n-1, \ldots, 2$}
    \State $r_i = \left|p_{i + 1} - p_i\right|$
    \State $\lambda_i = d_i / r_i$
    \State $p_i = (1 - \lambda_i) p_{i + 1} + \lambda_i p_i$
\EndFor
\Statex
\Statex \texttt{/* Zpětné natahování */}
\For{$i = 2, \ldots, n$}
    \State $r_i = \left|p_i - p_{i - 1}\right|$
    \State $\lambda_i = d_i / r_i$
    \State $p_i = (1 - \lambda_i) p_{i - 1} + \lambda_i p_i$
\EndFor
\end{algorithmic}
\end{algorithm}

Algoritmus klouby nejdříve natahuje směrem k cíli a poté je zase přitáhne k neposunuté základně. Jediné co ho omezuje jsou délky segmentů, které jsou opravovány posunutím po přímce mezi klouby. Tento postup je opakován dokud se poslední kloub nedostane do přijatelné vzdálenosti od cíle, nebo neprovede maximální počet kroků. Pokud cíl není dosažitelný, tj. jeho vzdálenost od základny je vetší než součet délek mezi klouby, stačí ho provést jednou.

\begin{figure}[h]
  \centering\leavevmode
  \includegraphics[width=.90\linewidth]{./images/unconstrained.png}\vskip-0.5cm
  \medskip
  \caption{Systém s volnými klouby}
  \label{fig:non-constrained-system}
\end{figure}

%%%%%%%%%%%%%%%%%%%%%%%%%%%%%%%%%%%%%%%%%%%%%%%%%%%%%%%%%%%%%%%%%%%%%%%%%%%%%%%%
\section{Omezení kloubů}
\label{sec:omezeni}
Původní článek ukazuje možnou implementaci omezení pohybu kloubů s 3 stupni volnosti (DOF) pro použití na lidských modelech. Specializovaných rozšíření algoritmu FABRIK existuje mnoho, pro inspiraci byl použit nedávný FABRIK-R\cite{bib:fabrik-r} zaměřený právě na systém otočných kloubů. Ten při opravování pozic a rotací aktuální kloub promítá do možných rovin splňující omezení sousedních. Mějme navíc ještě posloupnost vektorů ve směru osy rotace kloubu $v$, jeden krok první části algoritmu vypadá takto:

\begin{algorithm}[H]
\caption{Omezení kloubu na 1 DOF}\label{alg:1dof}
\begin{algorithmic}[1]
\Statex \texttt{/* Náprava podle $p_{i+1}$ */}
\State $\Phi_{prev} \gets$ rovina procházející $p_{i+1}$ definovaná rotačním vektorem $\vec{v}_{i+1}$
\State $\hat{p_i} \gets p_i$ promítnuté do $\Phi_{prev}$ se vzdáleností $d_i$ od $p_{i+1}$
\State Srovnej rotaci kloubu $\hat{p_i}$ ku $p_{i+1}$
\Statex
\Statex \texttt{/* Náprava podle $p_{i-1}$ */}
\State $\vec{n}_{\Phi_i} = \overrightarrow{p_{i-1}p_{i+1}} \times \vec{v}_{i-1}$
\State $\Phi_{i} \gets$ rovina obsahující $p_{i-1}$ a $p_{i+1}$ s normálovým vektorem $\vec{n}_{\Phi_i}$ 
\State Promítni $\vec{n}_{\Phi_i}$ do roviny $\Phi_{prev}$ a spočítej úhel $\theta$ mezi ním a rotačním vektorem $\hat{p_i}$
\State S použitím kvaternionů orotuj $\hat{p_i}$ kolem $p_{i+1}$ v ose $\vec{v}_{i+1}$ o úhel $\theta$
\State $p_i = \hat{p_i}$
\end{algorithmic}
\end{algorithm}

Promítání kloubu do roviny $\Phi_{prev}$ nám zaručí splnění omezení pozice a vzdálenosti od předchozího kloubu. $\Phi_{i}$ potom značí rovinu, ve které se musí aktuální kloub nacházet, aby splnil i omezení následujícího. Úhel $\theta$ potřebujeme právě proto, abychom kloub dostali do pozice na průniku těchto rovin. Úloha nemusí mít pouze jedno, nebo vůbec nějaké řešení. Tento postup pro jednoduchost předpokládá za sebou klouby, s ortogonálními vektory rotace. FABRIK-R\cite{bib:fabrik-r} ukazuje hledání $\theta$ pomocí řešení odvozené nelineární rovnice, kterému se tato metoda vyhýbá. Tento postup se analogicky provádí i při zpětném narovnávání v druhé části algoritmu.

\begin{figure}[h]
  \centering\leavevmode
  \includegraphics[width=.90\linewidth]{./images/6dof.png}\vskip-0.5cm
  \medskip
  \caption{Klouby omezené na 1 DOF}
  \label{fig:constrained-system}
\end{figure}

%%%%%%%%%%%%%%%%%%%%%%%%%%%%%%%%%%%%%%%%%%%%%%%%%%%%%%%%%%%%%%%%%%%%%%%%%%%%%%%%
\section{Testování}
\label{sec:testovani}
Algoritmus, tak jak je implementován, není perfektní. Cíle s tolerancí 1 cm dosáhlo rameno s 6 klouby o celkové délce 1.5 m v 86.5\% z 400 testů. Testované body byly rozmístěny v kouli kolem základny ve vzdálenosti 0.5 - 1.25 m. Z pozorovaní při interakci je patrné, že se algoritmus občas v závislosti na konfiguraci zasekává v lokáním minimu (obr. \ref{fig:locked-configuration}). V provedených testech algoritmus konvergoval v průměru po 12.86 iteracích.

\begin{figure}[h]
  \centering\leavevmode
  \includegraphics[width=.90\linewidth]{./images/locked.png}\vskip-0.5cm
  \medskip
  \caption{Zaseknutá konfigurace}
  \label{fig:locked-configuration}
\end{figure}

%%%%%%%%%%%%%%%%%%%%%%%%%%%%%%%%%%%%%%%%%%%%%%%%%%%%%%%%%%%%%%%%%%%%%%%%%%%%%%%%
\section{Aplikace}
\label{sec:aplikace}
Aplikace vytvořená v Unity\cite{bib:unity} zobrazuje robotickou paži s 6 DOF. Uživatel se kolem ramene může pohybovat a volně přesouvat cílový bod. Inverzní kinematika je řešena každý snímek, pokud je potřeba. Pro prezentaci je připraveno několik scén:
\begin{enumerate}
  \item FABRIK na volné soustavě
  \item Rozšířený FABRIK na rameni o 3 kloubech
  \item Paže s 6 klouby
\end{enumerate}

\begin{figure}[h]
  \centering\leavevmode
  \includegraphics[width=.90\linewidth]{./images/3dof.png}\vskip-0.5cm
  \medskip
  \caption{Paže s 3 DOF}
  \label{fig:second-scene}
\end{figure}

%%%%%%%%%%%%%%%%%%%%%%%%%%%%%%%%%%%%%%%%%%%%%%%%%%%%%%%%%%%%%%%%%%%%%%%%%%%%%%%%
\section{Závěr}
\label{sec:zaver}
Vytvořené řešení si zachovává rychlostní výhodu oproti jiným metodám, pro reálné použití by muselo být rozšířeno o vyhýbání se kolizím\cite{bib:fabrik-obstacle-avoidance} a řešení problému lokálních minim, např. pomocí nějakého přístupu simulovaného žíhání. Metody využívající FABRIK mají také problém s ovládáním orientace efektoru v cíli, přístupy ale existují\cite{bib:fabrik-effector-orientation}. Vizualizace metody ukazuje proof of concept, na kterém může být dále stavěno.

%%%%%%%%%%%%%%%%%%%%%%%%%%%%%%%%%%%%%%%%%%%%%%%%%%%%%%%%%%%%%%%%%%%%%%%%%%%%%%%%
% --- Bibliography
\nocite{bib:fabrik}
\nocite{bib:fabrik-r}
\nocite{bib:unity}
\nocite{bib:fabrik-obstacle-avoidance}
\nocite{bib:fabrik-effector-orientation}
%\bibliographystyle{plain-cz-online}
\bibliography{reference}

\end{document}
